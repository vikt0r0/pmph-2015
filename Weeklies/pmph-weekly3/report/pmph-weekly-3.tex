% !TEX TS-program = pdflatex
\documentclass[11pt,a4paper,english]{article}
\usepackage[ansinew]{inputenc}
\usepackage[T1]{fontenc}
\usepackage[obeyspaces, hyphens]{url}
\usepackage[top=4cm, bottom=4cm, left=3cm, right=3cm]{geometry}
\usepackage{enumerate}
\usepackage{amsmath}
\usepackage[pdftex]{graphicx}
\usepackage{mdwlist}
\usepackage{hyperref}
\usepackage{fancyhdr}
\usepackage{cite}
\usepackage{amsmath}
\usepackage{ulem}
\usepackage{babel} 
\usepackage{fancyvrb}
\usepackage{verbatimbox}
\usepackage{amsfonts}
\usepackage{amsthm}
\usepackage{minted}
\usepackage{color}
\usepackage{csquotes}
\usepackage{listings}
\usepackage{graphicx}
\usepackage{caption}
\usepackage{subcaption}
\usepackage{booktabs}
\usepackage{array}
\newcolumntype{P}[1]{>{\centering\arraybackslash}p{#1}}
	
\newcommand*\justify{%
  \fontdimen2\font=0.4em% interword space
  \fontdimen3\font=0.2em% interword stretch
  \fontdimen4\font=0.1em% interword shrink
  \fontdimen7\font=0.1em% extra space
  \hyphenchar\font=`\-% allowing hyphenation
}

\lstset{
    frame=lrtb,
    captionpos=b,
    belowskip=0pt
}

\captionsetup[listing]{aboveskip=5pt,belowskip=\baselineskip}

	
%\definecolor{lightgray}{rgb}{0.95,0.95,0.95}
%\renewcommand\listingscaption{Code}

\newcommand{\concat}{\ensuremath{+\!\!\!\!+\!\!}}

\pagestyle{fancy}
\headheight 35pt

\DefineVerbatimEnvironment{code}{Verbatim}{fontsize=\small}
\DefineVerbatimEnvironment{example}{Verbatim}{fontsize=\small}
\newcommand{\ignore}[1]{}

\hyphenation{character-ised}

\rhead{Assignment 3}
\lhead{PMPH}
\begin{document}

\thispagestyle{empty} %fjerner sidetal 
\hspace{6cm} \vspace{6cm}
\begin{center}
\textbf{\Huge {Programming Massively Parallel Hardware}}\\ \vspace{0.5cm}
\Large{Assignment 3}
\end{center}
\vspace{3cm}
\begin{center}
\Large{\textbf{Viktor Hansen}}
\end{center}
\vspace{6.0cm}
\thispagestyle{empty}

\newpage

\section*{Task 1 - Matrix Transposition}
The solutions are implemented in the \texttt{code/task1} directory. To compile and run the tests, invoke \texttt{make} followed by \texttt{make run} from the \texttt{code/task1} directory. The same is the case for the rest of the implemented exercises in this assignment.

\section*{Task 2 - Usefulness of Matrix Transposition}
\subsection*{2.a - Loop level parallelism}
The outer loop is not parallel as the the \texttt{accum} is declared outside the loop and needs to be privatized to preserve the semantics. Rewriting the code yields:

\captionsetup{justification=centering,margin=2cm}
\begin{minted}{text}
1. for i from 0 to N-1 // outer loop
2.   accum[i] = A[i,0] * A[i,0];
3.   B[i,0] = accum;
4.   for j from 1 to 63 // inner loop
5.     tmpA   = A[i, j];
6.     accum  = sqrt(accum) + tmpA*tmpA;
7.     B[i,j] = accum;
\end{minted}
\captionof{listing}{Code with \texttt{accum} privatized.\label{lst:sgmscanhost}}

There are no loop-carried dependencies in the inner loops, and thus it can be parallelized. Expressing it in terms of parallel operators yields \texttt{scanInc (\textbackslash a e -> sqrt(a) + e*e) accum}. Rewriting line 6 as \texttt{accum = accum + tmpA*tmpA} means that the inner loop can be rewritten as a map/scan (or segmented scan) composition. The Haskell code for the inner loop would look like \texttt{scanInc (+) accum \$ map (\textbackslash x->x*x)}.


\section*{Task 3 - Matrix Multiplication}
\subsection*{3.a - Sequential Implementation}
The sequential implementation is to be found in \texttt{code/matrix.cu}.


\subsection*{3.b - OMP Implementation}
Not implemented.

\subsection*{3.c - Naive CUDA Implementation}
Implemented in \texttt{code/matrix.cu}.

\subsection*{3.d - Tiled CUDA Implementation}
Attempted implementation of kernel in \texttt{code/matrix\_kernels.cu.h}, but I did not succeed. I attached code where I attempted to solve this about three weeks ago, attached in \texttt{code/matrixMult}. Unfortunately it only yields the correct results when the size of the matrix is a multiple of the block size. This can be run by invoking \texttt{make run} and running the generated binary.



%\bibliographystyle{plain}
%\bibliography{references}

%\section*{Appendix A}
%\subsection*{\path{src/commented-disassembly.txt}}
%\inputminted{text}{src/commented-disassembly.txt}

%\pagebreak

%\subsection*{\path{src/exploit.c}}
%\inputminted{c}{src/exploit.c}

\end{document}
